\documentclass[10pt,letterpaper]{article}
%\usepackage{graphicx}

\oddsidemargin=0in
\evensidemargin=0in
\topmargin=0in
\headheight=0in
\headsep=0in
\marginparwidth=0in
\marginparsep=0in
\raggedbottom
\textwidth=6.5in
\textheight=8.5in

\newcommand{\mdaload}{\texttt{mda-load}}
\newcommand{\version}{{v1.4.2}}

\title{ \vspace{-5ex} \texttt{mda-load} library \version\ manual}
\author{Dohn Alexander Arms \\ \texttt{dohnarms@anl.gov}}

\begin{document}
\maketitle


This library is used to load MDA files created by the
\textbf{saveData} program that is part of \textbf{EPICS}.
It will load files with MDA versions of 1.2, 1.3, and 1.4,
which are the only versions used currently.


\section{Compiling}

Compiling requires that you specify the \textbf{mda-load} library.
This can be done directly by specifying the filename
\verb+libmda-load.a+ and its path. If the library is in a standard
search location for libraries, you can use the ``\verb+-l mda-load+''
option with \textbf{gcc} or \textbf{ld}.


\section{Code}

An MDA file structure has a defined structure that is mimicked by the
data structures this library uses.  There first is a header describing
the scan(s), followed by the highest dimensional scan, then all the
lower dimensional scans, and finally by an optional section of extra
PVs.


\subsection{Functions}

There are functions for loading the entire file or just parts of it.
The function and structure definitions are in the include file
\verb+mda-load.h+.

In the following functions, \verb+fptr+ is a \verb+FILE+ pointer
corresponding to an open MDA file, normally gotten from \verb+fopen()+
and removed by \verb+fclose()+.  It doesn't matter what position the
pointer is at in the file when calling these functions.

\vspace{1.5ex}
\noindent
\verb+struct mda_file *mda_load( FILE *fptr)+ \linebreak
This function loads the entire MDA file at once, returning a pointer
to an \verb+mda_file+ structure.

\vspace{1.5ex}
\noindent
\verb+void mda_unload( struct mda_file *mda)+ \linebreak 
This will free the memory occupied by the \verb+mda_file+ to which
\verb+mda+ points.

\vspace{1.5ex}
\noindent
\verb+struct mda_header *mda_header_load( FILE *fptr)+ \linebreak
This function loads the MDA file header, returning a pointer to a
\verb+mda_header+ structure.

\vspace{1.5ex}
\noindent
\verb+void mda_header_unload( struct mda_header *header)+ \linebreak
This will free the memory occupied by the \verb+mda_header+ to which
\verb+header+ points.

\vspace{1.5ex}
\noindent
\verb+struct scan *mda_scan_load( FILE *fptr)+ \linebreak
This function loads the entire scan structure (all dimensions) for an
MDA file, returning a pointer to a \verb+mda_scan+ structure.  It is
the same as \verb+mda_subscan_load()+, where \verb+depth+ is zero and
\verb+recursive+ is one.

%\newpage

\vspace{1.5ex}
\noindent
\verb+struct mda_scan *mda_subscan_load( FILE *fptr, int depth, int *indices, int recursive)+ \linebreak
This function loads only a part of an MDA file's scans, returning a
pointer to the resulting \verb+mda_scan+ structure.  The \verb+depth+
refers to how many dimensions down into the scan the subscan is
located (it has to be less than the number of dimensions in the file),
and to the number of members in \verb+indices+.  The \verb+indices+
array contains the location of the subscan, starting with the highest
dimensional index; if \verb+depth+ is zero, \verb+indices+ can be set
to \verb+NULL+.  The \verb+recursive+ flag determines whether the
lower dimensional scans will also be read (if any exist).

\vspace{1.5ex}
\noindent
\verb+void mda_scan_unload( struct mda_scan *scan)+ \linebreak
This will free the memory occupied by the \verb+mda_scan+ to which
\verb+scan+ points.

\vspace{1.5ex}
\noindent
\verb+struct mda_extra *mda_extra_load( FILE *fptr)+ \linebreak
This function attempts to load the extra PVs for an MDA file. If they
exist, it returns a pointer to an \verb+mda_extra+ structure,
otherwise it returns \verb+NULL+.

\vspace{1.5ex}
\noindent
\verb+void mda_extra_unload( struct mda_extra *extra)+ \linebreak
This will free the memory occupied by the \verb+mda_extra+ to which
\verb+extra+ points. It is safe to call this function if \verb+extra+
is set to \verb+NULL+.

\vspace{1.5ex}
\noindent
\verb+struct mda_file *mda_update( FILE *fptr, struct mda_file *previous_mda)+ \linebreak
This function is the same as \verb+mda_load()+, except that it will
attempt to use the data from a previous \verb+mda_load()+ or
\verb+mda_update()+ to reduce the amount of data that needs to be
read.  This is intended for use when an MDA file is being read while
the scan is in progress, allowing only the newly written part of the
file to be read, reducing the CPU load.  The \verb+previous_data+
pointer that is passed needs to be either point to a valid
\verb+struct mda_struct+ or is NULL.  If \verb+previous_data+ is not
\verb+NULL+, its value after the function call is invalid, even if the
function returned a \verb+NULL+ due to an error; the data to which it
pointed was either taken by the newly returned
\verb+struct mda_struct+ or was released from memory.

\subsection{Example}

The simplest way to use this library is to simply load the entire MDA
file, using the template below.  Then there are only two functions
needed, \verb+mda_load()+ and \verb+mda_unload()+, with the rest of
the work coming from accessing the MDA data structure.

\begin{verbatim}
#include "mda-load.h"
...
FILE *fptr;
struct mda_file *mda;
...
if( (fptr = fopen(filename, "r")) == NULL)
    exit(1);
if( (mda = mda_load(fptr)) == NULL)
    exit(1);
fclose(fptr);
...
/* Access the mda structure here. */
...
mda_unload(mda);
...
\end{verbatim}

%\newpage

\section{Data}

Once you have a pointer to the allocated \verb+mda_file+, you can
access its members directly.  The definition of the various data
structures are in \verb+mda-load.h+.  Assuming that the data pointer
is called \verb+mda+, here's how you access the data.

The code takes advantage of the C99 integer data types in
\verb+stdint.h+, which allows explicitly declaring the size of an
integer.  Any of the \verb+int8_t+, \verb+int16_t+, or \verb+int32_t+
types can simply be used as an \verb+int+ if it makes your code
simpler (such as using \verb+%i+ with printf).


\subsection{Header}
\label{header}
\begin{verbatim}
(struct mda_header *) mda->header
      (float)   header->version
      (int32_t) header->scan_number
      (int16_t) header->data_rank
      (int32_t) header->dimensions[n] , [n] = [0] to [data_rank - 1]
      (int16_t) header->regular
      (int32_t) header->extra_pvs_offset
\end{verbatim}

This section contains the global data values for the MDA file.
\verb+version+ signifies the MDA format version, normally 1.3.
\verb+scan_number+ is the number assigned by \textbf{saveData} to the
scan.  \verb+data_rank+ show the number of dimensions to the scan (for
a 3-D scan, this is 3).  The \verb+dimensions+ array (with
\verb+data_rank+ elements) contains the number of elements for each
dimension of the scan, starting with the highest dimensional scan; for
a 4-D scan, \verb+dimensions[1]+ is the number of elements to the 3-D
scans.  \verb+regular+ signifies whether the dimensions of any of the
scans were changed while the overall scan was running; a nonzero value
signifies that the dimensions array is not totally correct, and the
number of requested points needs to be checked in each
scan. \verb+extra_pvs_offset+ gives, for the section of extra PV's,
the offset in bytes from the beginning of the file; if that section
does not exist, this will be \verb+0+.


\subsection{Scans}

\begin{verbatim}
(struct mda_scan *) mda->scan
      (int16_t)   scan->scan_rank
      (int32_t)   scan->requested_points
      (int32_t)   scan->last_point
      (int32_t *) scan->offsets
      (char *)    scan->name
      (char *)    scan->time
      (int16_t)   scan->number_positioners
      (int16_t)   scan->number_detectors
      (int16_t)   scan->number_triggers
      (struct mda_positioner *) scan->positioners[n] , 
                                [n] = [0] to [scan->number_positioners - 1]
            (int16_t) positioners[n]->number
            (char *)  positioners[n]->name
            (char *)  positioners[n]->description
            (char *)  positioners[n]->step_mode
            (char *)  positioners[n]->unit
            (char *)  positioners[n]->readback_name
            (char *)  positioners[n]->readback_description
            (char *)  positioners[n]->readback_unit
      (struct mda_detector * scan->detectors[n] ,
                                [n] = [0] to [scan->number_detectors - 1]
            (int16_t) detectors[n]->number
            (char *)  detectors[n]->name
            (char *)  detectors[n]->description
            (char *)  detectors[n]->unit
      (struct mda_trigger *) scan->triggers[n] ,
                                [n] = [0] to [scan->numbers_triggers - 1]
            (int16_t) triggers[n]->number
            (char *)  triggers[n]->name
            (float)   triggers[n]->command
      (double *) scan->positioners_data[n] , [n] = [0] to [scan->number_positioners - 1]
            (double) scan->positioners_data[n][m] ,
                                [m] = [0] to [scan->requested_points - 1]
      (float *) scan->detectors_data[n] , [n] = [0] to [scan->number_detectors - 1]
            (float) scan->detectors_data[n][m] , 
                                [m] = [0] to [scan->requested_points - 1]
      (struct mda_scan **) scan->sub_scans
\end{verbatim}

This section includes the scan data.  It is also recursive in nature
due to it being able to handle arbitrary dimensions.  

\subsubsection{Structure}
\label{scanure}

The overall structure for multidimensional files is dictated by
\verb+scan_rank+ and \verb+sub_scans+.  As long as \verb+scan_rank+ is
greater than one, \verb+sub_scans+ will not be \verb+NULL+ and will
contain an array of the next lower dimensional scans (it will be
\verb+NULL+ if \verb+mda_subscan_load()+ was used to retrieve the scan
when the \verb+recursive+ parameter set to zero).  For a
multidimensional scan, this takes the form of a tree, since each
sub-scan can also have its own sub-scans.  For a higher dimensional
scan, the values for the positioners and detectors apply to all scan
with its \verb+sub_scans+.

Suppose a 5$\times$8$\times$20 scan, where you want to access the
$(3,7,x)$ 1-D scan, you would access it as
\verb+mda->scan->sub_scans[2]->sub_scans[6]+.  However, if the scan
was aborted, \verb+mda->scan->sub_scans[2]+ or
\verb+mda->scan->sub_scans[2]->sub_scans[6]+ \textit{might} be
\verb+NULL+, depending on where the scan was aborted.  Using
\verb+last_point+ can let you know the last ``officially valid''
sub-scan is \verb+sub_scans[last_point - 1]+.  The reason that I say
``officially valid'' is that another scan \textit{might exist} at
\verb+sub_scans[last_point]+, as it was the scan in progress that was
aborted; one can use this data, but should take care.

\subsubsection{Variables}
\label{scan_variables}

As described before, \verb+scan_rank+ is the dimensionality of this
scan.  \verb+requested_points+ is how many points were wanted, while
\verb+last_point+ tells how many actually were finished.
\verb+offsets+ is an array of \verb+requested_points+ members, showing
the distance from the beginning of the MDA file to the subscans; if
the value is zero, then that scan does not exist.  \verb+name+ is the
name of the scanner in \textbf{EPICS}, while \verb+time+ is when this
particular scan was started.

\verb+number_positioners+ tells how many positioners are moved as part
of this scan.  The \verb+positioners+ array, holding
\verb+number_positioners+ members, has a description of each
positioner and its readback.  \verb+number+ is the internal number the
\textbf{scanRecord} uses to identify this positioner, while
\verb+name+ is what its called, and \verb+description+ describes it.
\verb+step_mode+ is how the scan determined what step to use: it can
be \textsl{linear}, where the spacing between steps is equal;
\textsl{table}, where the step positions are read from an array; or
\textsl{fly}, where the step positions are read back during an
on-the-fly scan. \verb+unit+ is the associated unit of the positioner.
Similarly, for the readback, there is \verb+readback_name+,
\verb+readback_description+, and \verb+readback_unit+.

The detector information is very similar to the positioners, as there
is a \verb+detectors+ array with \verb+number_detectors+ elements.
For each detectors, there is also a \verb+number+, \verb+name+,
\verb+description+, and \verb+unit+.

The trigger information is again similar to the positioners, with a
\verb+triggers+ array with \verb+number_triggers+ elements.  Each
trigger has a \verb+number+ and \verb+name+ associated to it, as well
as a \verb+command+, which is a value sent to \verb+name+ to trigger.

The positioner data values are held in an two dimensional array named
\verb+positioners_data+.  Since one can't allocate a two dimensional
array directly, it's actually an array of pointers (corresponding to
each detector), pointing to arrays of more pointers (corresponding the
the data).  To access the 8th data point of the 12th detector, one
would type \verb+(scan->positioners_data[11])[7]+; the parentheses are
not optional.

The detector data values, in \verb+detectors_data+, is accessed
similarly to the positioner data values.

The \verb+sub_scans+ variable is used for accessing lower dimensional
scans (if they exist).  It's described in Sec.~\ref{scanure}.


\subsection{Extra PV's}

\begin{verbatim}
(struct mda_extra *) mda->extra
      (int16_t) extra->number_pvs
      (struct mda_pv *) extra->pvs[n] , [n] = [0] to [number_pvs - 1]
            (char *)  pvs[n]->name
            (char *)  pvs[n]->description
            (int16_t) pvs[n]->type    
            (int16_t) pvs[n]->count
            (char *)  pvs[n]->unit
            (void *)  pvs[n]->values
\end{verbatim}

This section, which doesn't always exist (signified by \verb+extra+
being \verb+NULL+), contains extra PV's recorded during the scan.
\verb+number_pvs+ is the number of PV's contained, with the PV's being
held in an array \verb+pvs+.

For each PV, there is the \verb+name+ string and \verb+description+
string.  \verb+type+ lets you know what kind of data type it is, with
the correspondence seen in Table~\ref{pv_type_table}.  If \verb+type+
isn't \verb+EXTRA_STRING+, \verb+count+ gives the number of elements to
the array and \verb+unit+ string gives the unit for the values.  The
values themselves are held in an array \verb+values+.

\begin{table}[h]
  \caption{Extra PV data type}
  \centerline{\begin{tabular}{|c|c||c|l|}
      \hline
      \texttt{type} name & \texttt{type} value & C type & Description\\
      \hline\hline
      \texttt{EXTRA\char95PV\char95STRING} & \texttt{0}  & \texttt{(char *)}          & zero-terminated string \\
      \texttt{EXTRA\char95PV\char95INT8}   & \texttt{32} & \texttt{(int8\char95t *)}  & 8-bit integer array \\
      \texttt{EXTRA\char95PV\char95INT16}  & \texttt{29} & \texttt{(int16\char95t *)} & 16-bit integer array \\
      \texttt{EXTRA\char95PV\char95INT32}  & \texttt{33} & \texttt{(int32\char95t *)} & 32-bit integer array \\
      \texttt{EXTRA\char95PV\char95FLOAT}  & \texttt{30} & \texttt{(float *)}         & floating-point array\\
      \texttt{EXTRA\char95PV\char95DOUBLE} & \texttt{34} & \texttt{(double *)}        & double-precision floating-point array \\
      \hline
    \end{tabular}}
  \label{pv_type_table}
\end{table}

Accessing the values is done by setting pointer \verb+values+ to the
correct type, according to \verb+type+ and Table~\ref{pv_type_table}.
Suppose the third extra PV was of type \verb+EXTRA_DOUBLE+, and you
wanted to access its fifth member, this could be done using
\verb+((double *) pvs[2]->values)[4]+.

%\newpage

\section{Basic Information Routines}

It is possible to load only the basic information of an MDA file,
saving load time and memory.  This information includes everything in
the header, as well as the detector, positioner, and trigger
descriptions for each scan dimension. As this information is taken
from the first scan of each dimension, there is an assumption that
every other scan is correctly represented by the first scan of its
dimensionality. This also means that for a multidimensional file, not
all of the file is checked for errors; this can be done with
\verb+mda_test()+, which will cost some additional time.

\subsection{Functions}

\vspace{1.5ex}
\noindent
\verb+int mda_test( FILE *fptr)+ \linebreak
This function will check a file's integrity by loading the entire file
structure (but not keeping all the information in memory to keep
memory usage minimal), using the same error checking as
\verb+mda_load()+.  A returned value of \verb+0+ means that there was
no error, while \verb+1+ means there was an error.

\vspace{1.5ex}
\noindent
\verb+struct mda_fileinfo *mda_info_load( FILE *fptr)+ \linebreak This
function loads a fileinfo of the MDA file, returning a pointer to an
\verb+mda_fileinfo+ structure.  It does not check the validity of the
entire file, which can be done with \verb+mda_test()+.

\vspace{1.5ex}
\noindent
\verb+void mda_info_unload( struct mda_fileinfo *fileinfo)+ \linebreak 
This will free the memory occupied by the \verb+mda_fileinfo+ to which
\verb+fileinfo+ points.

%\newpage

\subsection{Structure}

\begin{verbatim}
(struct mda_fileinfo *) fileinfo
      (float)   fileinfo->version
      (int32_t) fileinfo->scan_number
      (int16_t) fileinfo->data_rank
      (int32_t) fileinfo->dimensions[n] , [n] = [0] to [data_rank - 1]
      (int16_t) fileinfo->regular
      (int32_t) fileinfo->last_topdim_point
      (char *)  fileinfo->time
      (struct mda_scaninfo *) fileinfo->scaninfos[n] , [n] = [0] to [data_rank - 1]
            (int16_t) scaninfos[n]->scan_rank
            (int32_t) scaninfos[n]->requested_points
            (char *)  scaninfos[n]->name
            (int16_t) scaninfos[n]->number_positioners
            (int16_t) scaninfos[n]->number_detectors
            (int16_t) scaninfos[n]->number_triggers
            (struct mda_positioner *) scaninfos[n]->positioners[n] , 
                                      [n] = [0] to [scan->number_positioners - 1]
                  (int16_t) positioners[n]->number
                  (char *)  positioners[n]->name
                  (char *)  positioners[n]->description
                  (char *)  positioners[n]->step_mode
                  (char *)  positioners[n]->unit
                  (char *)  positioners[n]->readback_name
                  (char *)  positioners[n]->readback_description
                  (char *)  positioners[n]->readback_unit
            (struct mda_detector * scaninfos[n]->detectors[n] ,
                                      [n] = [0] to [scan->number_detectors - 1]
                  (int16_t) detectors[n]->number
                  (char *)  detectors[n]->name
                  (char *)  detectors[n]->description
                  (char *)  detectors[n]->unit
            (struct mda_trigger *) scaninfos[n]->triggers[n] ,
                                      [n] = [0] to [scan->numbers_triggers - 1]
                  (int16_t) triggers[n]->number
                  (char *)  triggers[n]->name
                  (float)   triggers[n]->command
\end{verbatim}

The structure of the \verb+mda_fileinfo+ structure is nonrecursive,
and is thus easier to access. The variables \verb+version+,
\verb+scan_number+, \verb+data_rank+, \verb+dimensions+, and
\verb+regular+ are the same as described in Section~\ref{header}. The
variable \verb+last_topdim_point+ is the number of completed scans in
the highest dimensional scan, while \verb+time+ is the data and time
of the start of the overall measurement. The information for each
dimension's scans are in the \verb+scaninfos+ array.  The variables
contained in each entry is a subset of those found in the earlier
describes \verb+scan+, and all are described in in
Section~\ref{scan_variables}.



\section{Library Changes}

From 1.1 to 1.2, integers use types that denote how many bits they
use: \verb+char+ (as an 8-bit integer) $\rightarrow$ \verb+int8_t+,
\verb+short+ $\rightarrow$ \verb+int16_t+, and \verb+long+
$\rightarrow$ \verb+int32_t+.  The most likely change needed in user
code for this is with \verb+printf()+, where one would use \verb+%d+
instead of \verb+%ld+, etc.  The Extra PV data type names were also
renamed, to have them start with \verb+EXTRA_PV_+ and have the integer
length in the relevant names.


\end{document}


